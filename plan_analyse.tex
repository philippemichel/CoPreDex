% Options for packages loaded elsewhere
\PassOptionsToPackage{unicode}{hyperref}
\PassOptionsToPackage{hyphens}{url}
%
\documentclass[
]{article}
\usepackage{amsmath,amssymb}
\usepackage{iftex}
\ifPDFTeX
  \usepackage[T1]{fontenc}
  \usepackage[utf8]{inputenc}
  \usepackage{textcomp} % provide euro and other symbols
\else % if luatex or xetex
  \usepackage{unicode-math} % this also loads fontspec
  \defaultfontfeatures{Scale=MatchLowercase}
  \defaultfontfeatures[\rmfamily]{Ligatures=TeX,Scale=1}
\fi
\usepackage{lmodern}
\ifPDFTeX\else
  % xetex/luatex font selection
\fi
% Use upquote if available, for straight quotes in verbatim environments
\IfFileExists{upquote.sty}{\usepackage{upquote}}{}
\IfFileExists{microtype.sty}{% use microtype if available
  \usepackage[]{microtype}
  \UseMicrotypeSet[protrusion]{basicmath} % disable protrusion for tt fonts
}{}
\makeatletter
\@ifundefined{KOMAClassName}{% if non-KOMA class
  \IfFileExists{parskip.sty}{%
    \usepackage{parskip}
  }{% else
    \setlength{\parindent}{0pt}
    \setlength{\parskip}{6pt plus 2pt minus 1pt}}
}{% if KOMA class
  \KOMAoptions{parskip=half}}
\makeatother
\usepackage{xcolor}
\usepackage[margin=1in]{geometry}
\usepackage{graphicx}
\makeatletter
\def\maxwidth{\ifdim\Gin@nat@width>\linewidth\linewidth\else\Gin@nat@width\fi}
\def\maxheight{\ifdim\Gin@nat@height>\textheight\textheight\else\Gin@nat@height\fi}
\makeatother
% Scale images if necessary, so that they will not overflow the page
% margins by default, and it is still possible to overwrite the defaults
% using explicit options in \includegraphics[width, height, ...]{}
\setkeys{Gin}{width=\maxwidth,height=\maxheight,keepaspectratio}
% Set default figure placement to htbp
\makeatletter
\def\fps@figure{htbp}
\makeatother
\setlength{\emergencystretch}{3em} % prevent overfull lines
\providecommand{\tightlist}{%
  \setlength{\itemsep}{0pt}\setlength{\parskip}{0pt}}
\setcounter{secnumdepth}{-\maxdimen} % remove section numbering
\newlength{\cslhangindent}
\setlength{\cslhangindent}{1.5em}
\newlength{\csllabelwidth}
\setlength{\csllabelwidth}{3em}
\newlength{\cslentryspacingunit} % times entry-spacing
\setlength{\cslentryspacingunit}{\parskip}
\newenvironment{CSLReferences}[2] % #1 hanging-ident, #2 entry spacing
 {% don't indent paragraphs
  \setlength{\parindent}{0pt}
  % turn on hanging indent if param 1 is 1
  \ifodd #1
  \let\oldpar\par
  \def\par{\hangindent=\cslhangindent\oldpar}
  \fi
  % set entry spacing
  \setlength{\parskip}{#2\cslentryspacingunit}
 }%
 {}
\usepackage{calc}
\newcommand{\CSLBlock}[1]{#1\hfill\break}
\newcommand{\CSLLeftMargin}[1]{\parbox[t]{\csllabelwidth}{#1}}
\newcommand{\CSLRightInline}[1]{\parbox[t]{\linewidth - \csllabelwidth}{#1}\break}
\newcommand{\CSLIndent}[1]{\hspace{\cslhangindent}#1}
\usepackage{booktabs}
\usepackage{longtable}
\usepackage{array}
\usepackage{multirow}
\usepackage{wrapfig}
\usepackage{float}
\usepackage{colortbl}
\usepackage{pdflscape}
\usepackage{tabu}
\usepackage{threeparttable}
\usepackage{threeparttablex}
\usepackage[normalem]{ulem}
\usepackage{makecell}
\usepackage{xcolor}
\ifLuaTeX
  \usepackage{selnolig}  % disable illegal ligatures
\fi
\IfFileExists{bookmark.sty}{\usepackage{bookmark}}{\usepackage{hyperref}}
\IfFileExists{xurl.sty}{\usepackage{xurl}}{} % add URL line breaks if available
\urlstyle{same}
\hypersetup{
  pdftitle={COPREDEX},
  pdfauthor={Philippe MICHEL},
  hidelinks,
  pdfcreator={LaTeX via pandoc}}

\title{COPREDEX\thanks{E. Devaux}}
\usepackage{etoolbox}
\makeatletter
\providecommand{\subtitle}[1]{% add subtitle to \maketitle
  \apptocmd{\@title}{\par {\large #1 \par}}{}{}
}
\makeatother
\subtitle{Plan d'analyse - V3}
\author{Philippe MICHEL}
\date{22 février 2024}

\begin{document}
\maketitle

{
\setcounter{tocdepth}{2}
\tableofcontents
}
Ce document ne concerne que l'analyse statistique des données.

L'analyse sera réalisée en per protocole, plus pertinente pour une étude
de non-infériorité. Le risque \(\alpha\) retenu est de 0,05 \& la
puissance de 0,8.

Vu les effectifs prévus l'hypothèse de normalité ne peut pas a priori
être retenue \& des tests non paramétriques seront utilisés.

\hypertarget{description-de-la-population}{%
\subsection{description de la
population}\label{description-de-la-population}}

La description de la population concerne les données recueillies à
l'inclusion :

\begin{itemize}
\tightlist
\item
  Clinique
\item
  Biologie
\item
  Score EQ5D
\end{itemize}

Les données discrètes seront présentés en pourcentage puis comparées par
un test exact de Fisher. Les données numériques seront présentées par
leur médiane \& les quartiles puis comparées par le test non
paramétrique de Wilcoxon.

Une comparaison simple entre les deux groupes sera réalisée pour
vérifier l'absence de différence gênante : test du \(\chi 2\) pour les
données catégorielles \& test de Wilcoxon pour les données numériques
après vérification de l'égalité des variances.

Une comparaison simple entre les deux groupes sera réalisée pour
vérifier l'absence de différence gênante : test du \(\chi 2\) pour les
données catégorielles \& test de Student pour les données numériques
après vérification de l'égalité des variances.

\hypertarget{objectif-principal}{%
\subsection{Objectif principal}\label{objectif-principal}}

\textbf{Évaluation de la mortalité à J28}

La comparaison du nombre de patients décédés sera réalisée par un test
de \(\chi 2\) de Dunnett et Gent en unilatéral. Une comparaison simple
en intention de traiter sera réalisée par un test exact de Fisher.

La comparaison du nombre de patients décédés sera réalisée par un test
de \(\chi 2\) de Dunnett et Gent en unilatéral.

Une analyse des courbes de survie (Kaplan-Meier) sera réalisée avec
comparaison entre les deux groupes par la méthode du LogRank. Si
nécessaire un ajustement par un modèle de Cox sera réalisé (sur l'âge
?).

\hypertarget{objectifs-secondaires}{%
\subsection{Objectifs secondaires}\label{objectifs-secondaires}}

Le grand nombre de paramètres suivis entraîne un trop grand nombre de
tests vu la taille de la population. Des choix devront être fait.

\hypertarget{jours-doxyguxe9nothuxe9rapie}{%
\subsubsection{Jours
d'oxygénothérapie}\label{jours-doxyguxe9nothuxe9rapie}}

La comparaison des durées de recours à l'O2 entre les deux groupes sera
faite par un test de Wilcoxon-Mann-Whitney.

Un graphique d'évolution des SpO2 patient par patient (graphique * en
fagots*) sera tracé.

\hypertarget{transferts-en-ruxe9animation}{%
\subsubsection{Transferts en
réanimation}\label{transferts-en-ruxe9animation}}

La comparaison du nombre de transferts en réanimation entre les deux
groupes sera faite par un test de Wilcoxon-Mann-Whitney. Une étude par
courbes de survie (Kaplan-Meier) sera réalisée en prenant comme
évènement le décès ou le transfert en réanimation avec comparaison entre
les deux groupes par un test de Log-Rank après vérification des
conditions d'emploi.

La comparaison du nombre de transferts en réaniamtion entre les deux
groupes sera faite par un test de Wilcoxon-Mann-Whitney. Une étude par
courbes de survie (Kaplan-Meier) sera réalisée en prenant comme
évènement le décès ou le transfert en réanimation avec comparaison entre
les deux groupes par un test de Log-Rank.

\hypertarget{uxe9volution-de-la-qualituxe9-de-vie}{%
\subsubsection{Évolution de la qualité de
vie}\label{uxe9volution-de-la-qualituxe9-de-vie}}

Les évolutions du score EQ5D, de la mobilité, de l'autonomie, des
activités courantes, des douleurs ou de l'anxiété seront comparés entre
les deux groupes par des tests de \(\chi 2\) avec au besoin correction
de Yates.

\hypertarget{uxe9vuxe8nements-effets-induxe9sirables}{%
\subsubsection{Évènements \& effets
indésirables}\label{uxe9vuxe8nements-effets-induxe9sirables}}

Les effets indésirables seront listés. Une comparaison simple du nombre
d'EI entre les deux groupes sera réalisée par un test exact de Fisher.

\hypertarget{technique}{%
\subsection{Technique}\label{technique}}

L'analyse statistique sera réalisée avec le logiciel \textbf{R} (R Core
Team 2022) \& diverses librairies en particulier celles du
\texttt{tidyverse} (Wickham et al. 2019), \texttt{epiDisplay}
(Chongsuvivatwong 2022) \& \texttt{baseph} (Michel 2023).

Les effets indésirables seront listés. Une comparaison simple du nombre
d'EI entre les deux groupes sera réalisée par un test de \(\chi 2\) avec
au besoin correction de Yates.

\hypertarget{refs}{}
\begin{CSLReferences}{1}{0}
\leavevmode\vadjust pre{\hypertarget{ref-epid}{}}%
Chongsuvivatwong, Virasakdi. 2022. \emph{epiDisplay: Epidemiological
Data Display Package}.
\url{https://CRAN.R-project.org/package=epiDisplay}.

\leavevmode\vadjust pre{\hypertarget{ref-baseph}{}}%
Michel, Philippe. 2023. \emph{Baseph : Un Package Pour Les Études
Cliniques Simples}. Pontoise, France.
\url{https://github.com/philippemichel/baseph}.

\leavevmode\vadjust pre{\hypertarget{ref-rstat}{}}%
R Core Team. 2022. \emph{R: A Language and Environment for Statistical
Computing}. Vienna, Austria: R Foundation for Statistical Computing.
\url{https://www.R-project.org/}.

\leavevmode\vadjust pre{\hypertarget{ref-tidy}{}}%
Wickham, Hadley, Mara Averick, Jennifer Bryan, and al. 2019. {``Welcome
to the {tidyverse}.''} \emph{Journal of Open Source Software} 4 (43):
1686. \url{https://doi.org/10.21105/joss.01686}.

\end{CSLReferences}

\end{document}
