\hypertarget{copredex}{%
\section{COPREDEX}\label{copredex}}

\hypertarget{plan-danalyse---v2}{%
\subsection{Plan d'analyse - V2}\label{plan-danalyse---v2}}

\hypertarget{philippe-michel}{%
\subsubsection{Philippe MICHEL}\label{philippe-michel}}

\hypertarget{mai-2022}{%
\subsubsection{10 mai 2022}\label{mai-2022}}

\hypertarget{TOC}{}
\begin{itemize}
\tightlist
\item
  \protect\hyperlink{description-de-la-population}{description de la
  population}
\item
  \protect\hyperlink{objectif-principal}{Objectif principal}
\item
  \protect\hyperlink{objectifs-secondaires}{Objectifs secondaires}

  \begin{itemize}
  \tightlist
  \item
    \protect\hyperlink{jours-doxyguxe9nothuxe9rapie}{Jours
    d'oxygénothérapie}
  \item
    \protect\hyperlink{transferts-en-ruxe9animation}{Transferts en
    réanimation}
  \item
    \protect\hyperlink{uxe9volution-de-la-qualituxe9-de-vie}{Évolution
    de la qualité de vie}
  \item
    \protect\hyperlink{uxe9vuxe8nements-effets-induxe9sirables}{Évènements
    \& effets indésirables}
  \end{itemize}
\item
  \protect\hyperlink{technique}{Technique}
\end{itemize}

\hypertarget{content-wrapper}{}
\leavevmode\hypertarget{main-content}{}%
Ce document ne concerne que l'analyse statistique des données.

L'analyse sera réalisée en per protocole, plus pertinente pour une étude
de non-infériorité. Le risque
{\textbackslash(\textbackslash alpha\textbackslash)} retenu est de 0,05
\& la puissance de 0,8.

\hypertarget{description-de-la-population}{%
\subsection{description de la
population}\label{description-de-la-population}}

La description de la population concerne les données recueillies à
l'inclusion :

\begin{itemize}
\tightlist
\item
  Clinique
\item
  Biologie
\item
  Score EQ5D
\end{itemize}

Une comparaison simple entre les deux groupes sera réalisée pour
vérifier l'absence de différence gênante : test du
{\textbackslash(\textbackslash chi 2\textbackslash)} pour les données
catégorielles \& test de Student pour les données numériques après
vérification de l'égalité des variances.

\hypertarget{objectif-principal}{%
\subsection{Objectif principal}\label{objectif-principal}}

\textbf{Évaluation de la mortalité à J28}

La comparaison du nombre de patients décédés sera réalisée par un test
de {\textbackslash(\textbackslash chi 2\textbackslash)} de Dunnett et
Gent en unilatéral.

Une analyse des courbes de survie (Kaplan-Meier) sera réalisée avec
comparaison entre les deux groupes par la méthode du LogRank. Si
nécessaire un ajustement par un modèle de Cox sera réalisé (sur l'âge
?).

\hypertarget{objectifs-secondaires}{%
\subsection{Objectifs secondaires}\label{objectifs-secondaires}}

Le grand nombre de paramètres suivis entraîne un trop grand nombre de
tests vu la taille de la population. Des choix devront être fait.

\hypertarget{jours-doxyguxe9nothuxe9rapie}{%
\subsubsection{Jours
d'oxygénothérapie}\label{jours-doxyguxe9nothuxe9rapie}}

La comparaison des durées de recours à l'O\textsubscript{2} entre les
deux groupes sera faite par un test de Wilcoxon-Mann-Whitney.

Un graphique d'évolution des SpO\textsubscript{2} patient par patient
(graphique * en fagots*) sera tracé.

\hypertarget{transferts-en-ruxe9animation}{%
\subsubsection{Transferts en
réanimation}\label{transferts-en-ruxe9animation}}

La comparaison du nombre de transferts en réaniamtion entre les deux
groupes sera faite par un test de Wilcoxon-Mann-Whitney. Une étude par
courbes de survie (Kaplan-Meier) sera réalisée en prenant comme
évènement le décès ou le transfert en réanimation avec comparaison entre
les deux groupes par un test de Log-Rank.

\hypertarget{uxe9volution-de-la-qualituxe9-de-vie}{%
\subsubsection{Évolution de la qualité de
vie}\label{uxe9volution-de-la-qualituxe9-de-vie}}

Les évolutions du score EQ5D, de la mobilité, de l'autonomie, des
activités courantes, des douleurs ou de l'anxiété seront comparés entre
les deux groupes par des tests de {\textbackslash(\textbackslash chi
2\textbackslash)} avec au besoin correction de Yates.

\hypertarget{uxe9vuxe8nements-effets-induxe9sirables}{%
\subsubsection{Évènements \& effets
indésirables}\label{uxe9vuxe8nements-effets-induxe9sirables}}

Les effets indésirables seront listés. Une comparaison simple du nombre
d'EI entre les deux groupes sera réalisée par un test de
{\textbackslash(\textbackslash chi 2\textbackslash)} avec au besoin
correction de Yates.

\hypertarget{technique}{%
\subsection{Technique}\label{technique}}

L'analyse statistique sera réalisée avec le logiciel \textbf{R} {(R Core
Team 2022)} \& diverses librairies en particulier \texttt{tidyverse}
{(Wickham et al. 2019)} \& \texttt{epiDisplay} {(Chongsuvivatwong
2018)}.

\hypertarget{refs}{}
\begin{cslreferences}
\hypertarget{ref-epid}{}
Chongsuvivatwong, Virasakdi. 2018. \emph{epiDisplay: Epidemiological
Data Display Package}.
\url{https://CRAN.R-project.org/package=epiDisplay}.

\hypertarget{ref-rstat}{}
R Core Team. 2022. \emph{R: A Language and Environment for Statistical
Computing}. Vienna, Austria: R Foundation for Statistical Computing.
\url{https://www.R-project.org/}.

\hypertarget{ref-tidy}{}
Wickham, Hadley, Mara Averick, Jennifer Bryan, Winston Chang, Lucy
D'Agostino McGowan, Romain François, Garrett Grolemund, et al. 2019.
{``Welcome to the {tidyverse}.''} \emph{Journal of Open Source Software}
4 (43): 1686. \url{https://doi.org/10.21105/joss.01686}.
\end{cslreferences}
